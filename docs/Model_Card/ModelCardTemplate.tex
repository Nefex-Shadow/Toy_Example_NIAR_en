\documentclass{article}
\usepackage[utf8]{inputenc}


% Esse Template é uma adaptação do modelo feito por Christian Garbin:
% https://www.overleaf.com/latex/templates/model-card-template/fjmvzbbbxmwx
% Licenciada pelo Creative Commons CC BY 4.0

% Poucas adaptações foram feitas. Foram modificadas as disposições das tabelas,
% traduziu-se os textos e foi retirado o texto introdutório,
% a fim de facilitar o uso desse template.

% Para saber mais sobre o uso de Model Cards, segue-se o artigo que propôs essa ferramenta:
% https://arxiv.org/abs/1810.03993


% Usado no Template
\usepackage{setspace}
\usepackage{subcaption}
\usepackage{changepage} % Ajuste das margens
\usepackage[breakable]{tcolorbox}
\usepackage{float} % Para as tabelas de tcolorbox https://tex.stackexchange.com/a/274342
\usepackage{enumitem}
\usepackage{geometry}

\geometry{vmargin=30pt,hmargin=30pt}
\begin{document}


% Cada seção deve ser breve, em formato de lista
% Esse template faz a formatação das listas de foma compacta para facilitar manter
% o model card dentro da recomendação de uma a duas páginas.
\newenvironment{mcsection}[1]
    {%
        \textbf{#1}

        % Redução das margens para uso melhor do espaço
        \begin{itemize}[leftmargin=*,topsep=0pt,itemsep=-1ex,partopsep=1ex,parsep=1ex,after=\vspace{\medskipamount}]
    }
    {%
        \end{itemize}
    }

\begin{singlespace}

\tcbset{colback=white!10!white}

% Título do model card, alterar aqui
\begin{tcolorbox}[title=\textbf{Model Card - \textit{Model Name}},
    breakable, sharp corners, boxrule=0.7pt]

% Small: reduz tamanho da fonte para uma pequena, mas legível.
\small{


\begin{mcsection}{Detalhes do Modelo}
    \item Desenvolvedor, data
    \item Tipo de modelo, versão
    \item Motivações, objetivos
    \item Descrição
\end{mcsection}

\begin{mcsection}{Uso pretendido}
    \item Casos onde o modelo pode ser aplicado
    \item Casos onde o modelo não deve ser aplicado
\end{mcsection}

\begin{mcsection}{Fatores}
    \item Fatores que podem afetar a performance do modelo.
\end{mcsection}


\begin{mcsection}{Dados de Treinamento}
    \item Descrição dos dataset usado para treinamento
    \item Sumário do processamento dos dados (se houver)
\end{mcsection}

\begin{mcsection}{Dados de Avaliação}
    \item Descrição dos dataset usado para avaliação
    \item Sumário do processamento dos dados (se houver)
\end{mcsection}

\begin{mcsection}{Métricas}
    \item Métrica 1: Descrição
    \item Métrica 2: Descrição
    \item Métrica 3: Descrição
\end{mcsection}

\begin{mcsection}{Avisos e Recomendações}
    \item Avisos 1
    \item Avisos 2
    \item Recomendações 1
    \item Recomendações 2
\end{mcsection}

\centering
\textbf{Análise Quantitativa}
\vspace{5pt}
% Para retirar ou duplicar uma tabela,
% selecione desde begin{minipage} até end{minipage}.

\begin{minipage}{.25\textwidth}
\centering
\begin{tabular}{lrr}
\textbf{Métrica} & \textbf{Modelo}\\ \hline
Métrica 1 & 1234.5\\
Métrica 2 & 1234.5\\
Métrica 3 & 1234.5\\ \hline
\end{tabular}
\captionof{table}{Avaliação 1}
\end{minipage}


\begin{minipage}{.25\textwidth}
\centering
\begin{tabular}{lrr}
\textbf{Métrica} & \textbf{Modelo}\\ \hline
Métrica 1 & 1234.5\\
Métrica 2 & 1234.5\\
Métrica 3 & 1234.5\\ \hline
\end{tabular}
\captionof{table}{Avaliação 2}
\end{minipage}


\begin{minipage}{.25\textwidth}
\centering
\begin{tabular}{lrr}
\textbf{Métrica} & \textbf{Modelo}\\ \hline
Métrica 1 & 1234.5\\
Métrica 2 & 1234.5\\
Métrica 3 & 1234.5\\ \hline
\end{tabular}
\captionof{table}{Avaliação 3}
\end{minipage}
}
\end{tcolorbox}
\end{singlespace}

\end{document}
